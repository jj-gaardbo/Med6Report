\chapter{Methods}\label{chap:methods}
    This chapter will describe the approaches used to go from the final problem statement, to testing the final prototype. Mirroring the testing methods of the embedded design model paper\cite{embeddedDesignModel, embeddedDesignModelTestDetails}, as closely as possible. The initial design will be based on the design requirements listed in\autoref{designReq}.
    
    \section{Initial Design}
    To come up with a initial design idea, brainstorming in multiple iterations will provide an early design that can be improved upon. The idea is to first brainstorm as many ideas as possible and then present them. Then, these ideas can be combined and improved and potentially be made into a better idea. In the end, a few select ideas will be chosen and voted on to get the best one.

    \section{Usability test}
    Once the the design is finalised, a usability test will be conducted in order to make sure that the interactions work as intended.
    
    \subsection{Sampling}
    Convenience sampling allows for an effective gathering of participants, and will be used for usability testing\cite{bjoernerBog}.   
    
    \subsection{Observation}
    For the usability test, a Direct Observation method will be used\cite{bjoernerBog}. This is to observe the participants interactions within the game and to find out how they interact with an object. The purpose of this, is to find out if the participant knows how to interact or if somethings needs to be changed with the interaction.
    
    \subsection{System Usability Scale(SUS)}
    Along with the observations, the SUS will also be used to measure the usability of the prototype. SUS is a scale which consists of 10 questionnaire items with 5 responses ranging from Strongly Disagree to Strongly Agree. The results are gathered and given a SUS score which can be used to measure how good the usability in the prototype is\cite{SUS}. The questions can be found in appendix \ref{SUS}.
    
    
    \section{Final test}\label{methods:ft}
     \subsection{Sampling}
     For the final test, the probability sampling method; Simple Random Sampling will be used\cite{bjoernerBog}. Each person in the target group population will have an equal chance to be selected.
     
     The test will be done using two different prototypes and the participants have the same probability of getting picked to use either one. 
    
    \subsection{Test process}
    For the final test a semi-Structured interview will be used\cite{bjoernerBog}. A participant will be presented with either the intermixed version of the game or the overloaded one, they will not get the opportunity to try both as this might create a bias towards the results\cite{bjoernerBog}. This was done according to the test methods of Kaufman's paper\cite{embeddedDesignModel, embeddedDesignModelTestDetails}. The paper this project tries replicate inside Virtual Reality, focuses on a quantitative approach to testing, with more than 500 combined participants for their two different study methods, but as this project is limited in scope and resources, a more manageable sample size of 20 was chosen. With this smaller sample size, the testing method also need to change to a more qualitative one, to be viable for a conclusive test. Hence the choice of a semi-structured interview.
    
    \subsection{Analysing the data}
    The analysis of the data will be done using Traditional Coding\cite{bjoernerBog}. Traditional coding is a way to organise and analyse data accordingly. The coding follows four steps of processing; Organising, Recognising, Coding, and interpreting\cite{bjoernerBog}. 
    
    \textbf{Organising} is preparing the data, this could be transcripts of interviews, cataloguing the visual material, or making notes\cite{bjoernerBog}.
    
    \textbf{Recognising} looking at the data over and over again will give you an idea of major themes and re-occurrences expressed by the respondents\cite{bjoernerBog}. 
    
    \textbf{Coding} organising the data and putting it into categories and sub-categories with different labels\cite{bjoernerBog}.
    
    \textbf{Interpretation} analysing and interpreting the coded data, this will give the idea of what the outcome of the data is and what the meaning is\cite{bjoernerBog}.
    
    \subsection{Test setup}
    There is gonna be two setups where a participant will either take part of an intermixed version of the prototype, or an overloaded one. After playing the game, the participant will then be asked the questions from the interview. Expected time per test is 20 minutes; 5 minutes for setup, 5 for playing the game, and 10 for discussion. Total amount of participants at least expected, is 20(10 per prototype).