\chapter{Methods}\label{chap:methods}
    This chapter will describe the approaches used to go from the final problem statement, to testing the final prototype. The initial design will be based on the design requirements listed in\autoref{designReq}.
    
    \section{Initial Design}
    To come up with a initial design idea, brainstorming in multiple iterations will provide an early design that can be improved upon. The idea is to first brainstorm as many ideas as possible and then present them. Then, these ideas can be combined and improved and potentially be made into a better idea. In the end, a few select ideas will be chosen and voted on to get the best one.

    \section{Testing in VR}
    The following section will present a list of things that will be considered when testing specifically with VR.\\
    
     (https://uxstudioteam.com/ux-blog/vr-in-ux-testing/)\\
      1. The physical environment: Consider how much space the participant will need to operate in, and in which surroundings (open-air sessions)? Specific “mixed reality labs” offer spaces to run augmented and virtual reality user research sessions. They provide the kits (head-mounted displays like Oculus Rift and Microsoft Hololens) and a space to configure according to needs of the research.\\
    2. The technology: Really get to know what you will work with. Involve a developer or product designer in the process and work in a cross-functional team, which has many advantages. Since the technology is fairly new, it has no common interaction style yet, so you will be testing the software and hardware together. This can pose a challenge for both the participant and the research team.\\
    3. Safety – Remove all obstacles (cables, etc) so users can maneuver freely. \\
    4. The hardware – Make the headset comfortable even for people with glasses.\\ 5. Hygiene – VR experiences can get sweaty. Keep the headset clean between sessions.\\
    6. Recruitment – When planning recruitment for VR testing, the first thing you might ask is: do you want participants new to VR or not? Obviously, an individual’s digital literacy and experience in VR correlates to the time needed to onboard and get used to the experience. But don’t get bogged down at this point. These technologies are still emerging and many future users have yet to adopt. Instead, focus on other traits as with conventional user testing, for example, if the participant would likely become a user.\\
    7. Research plan: VR contains a physical and non-physical experience as well. Make it clear what you’re testing, even with the users. Are you testing their experience with the device setup or the in-app experience? What characteristics in their behavior are you observing? What are you recording and how? Set your research goals and define variables to observe and record well before starting.\\
    8. Privacy: Apps can collect data about users’ surroundings and their actions. As in non-VR studies, clarify what it collects and how it uses that data, especially if running tests in the user’s home.\\
    
    \textbf{II. VR user testing}\\
    1. Unfamiliarity with the technology: Leave enough preparation time for users to get familiar with the technology, headsets, and surroundings. Let them practice a few minutes before they dive into the “other world”. Help them feel safe throughout the test. Consider also in which order the participant will go through different experiences. Don’t “shock” them too early. As they become more familiar with the controls, their experience likely will improve automatically.\\
    
    2. Cyber-sickness: Prepare for situations where some participants might get nauseated. But don’t make a self-fulfilling prophecy; just tell them they can take a break at any time. Note symptoms and causes (such as frame rate, sudden acceleration, session length, standing versus seated position, the age of the participant) and do not proceed without getting them fixed. Take responsibility for your participants and don’t let them leave the room with a bad experience. In such cases, you need to have enough capacity that allows for session terminations, so it’s a good idea to over-recruit.\\
    
    3. Facilitation: Running VR sessions requires lot more vigilance than conventional user tests. Many more factors need attention and recording, such as: noting verbal and non-verbal clues while keeping an eye on what the user is looking at and how they interact. You also have to make sure they don’t walk into anything or get tangled up in cables. Facilitation also proves harder in VR because the user experiences an entirely constructed physical environment different from the context of the interviewer. The interviewer can’t immerse themselves in the environment at the same time, and a disembodied “interviewer” voice could ruin the experience. But asking questions to understand the user’s experience is unavoidable. Facilitation techniques can surely improve in this area, but some solutions might help reduce distraction, such as a facilitator within the same VR space or interactive avatars.\\
    
    4. Recording and note taking: The large headsets covering the users’ heads or subjects looking in another direction make it harder to read facial expressions and emotional responses in VR sessions. Additional observation tools such as cameras and more note taking participants can help catch all of the important feedback.\\
    
    5. Participants: A note taker can assume some of the facilitator’s workload. Leave enough space for them to move and follow the participant’s movements.\\
    
    \textbf{III. After the test}\\
    Post-interview: Give participants enough time to reflect on what they have seen. Ask more questions after the VR part of the session ends. At this point, fill out a Simulation Sickness Questionnaire and make sure participants feel OK physically.
    
    \section{Usability test}
    Once the the design is finalised, a usability test will be conducted in order to make sure that the interactions work as intended.
    
    \subsection{Sampling}
    Convenience sampling allows for an effective gather of participants, and will be used for usability testing\cite{bjoernerBog}.   
    
    \subsection{Observation}
    For the usability test, a Direct Observation method will be used\cite{bjoernerBog}. This is to observe the participants interactions within the game and to find out how they interact with an object. The purpose of this, is to find out if the participant knows how to interact or if somethings needs to be changed with the interaction.
    
    \subsection{System Usability Scale(SUS)}
    Along with the observations, the SUS will also be used to measure the usability of the prototype. SUS is a scale which consists of 10 questionnaire items with 5 responses ranging from Strongly Disagree to Strongly Agree. The results is gathered and given a SUS score which can be used to measure how good the usability in the prototype is\cite{SUS}. The questions can be found in appendix \ref{SUS}.
    
\textbf{Testing in VR:}
    (https://uxstudioteam.com/ux-blog/vr-in-ux-testing/)\\
    \textbf{I. Preparations:}\\
   
    
    \section{Final test}
     \subsection{Sampling}
     For the final test, the probability sampling method; Simple Random Sampling will be used\cite{bjoernerBog}. Each person in the target group population will have an equal chance to be selected. 
     
     The test will be done using two different prototypes and the participants have the same probability of getting picked to use either one. 
    
    \subsection{To test for behaviour change in regards to meat consumption}
    Semi-Structured interview\cite{bjoernerBog}. A participant will be presented with either the intermixed version of the game or the none-intermixed one, they will not get the opportunity to try both as this would create a bias towards the results.
    "What was your first impression of the game?"
    "What do you think the message of the game was?"
    "Would you consider eating less meat or buy more locally, after this experience?"
    "Did you feel that the of the game came on too strong?"
    "Did you feel like you got a new perspective on meat consumption? -How so and how did it change?"
    "Did you reflect on your current standings in regards to meat consumption?"
    "After trying this game, do you feel any cognitive dissonance?"
    
    \subsection{Analysing the data}
    The analysis of the data will be done using either Meaning Condensation or Traditional Coding\cite{bjoernerBog}. Meaning Condensation takes what the test participant says in the interview and condenses the meaning down to only a few words or sentences. The condensed meanings will then be analysed and compared to other participants answered. 
    
    \subsection{Test setup}
    There is gonna be two setups where a participant will either take part of an intermixed version of the prototype, or one without intermixing. After playing the game, the participant will then be asked the questions from the interview. Expected time per test is 20 minutes; 5 minutes for setup, 5 for playing the game, and 10 for discussion. Total amount of participants at least expected, is 20(10 per prototype), but as many as possible.