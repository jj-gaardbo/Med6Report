\chapter{Methods}\label{chap:methods}
    This chapter will describe the approaches used to go from the final problem statement, to testing the final prototype. The initial design will be based on the design requirements listed in\autoref{designReq}.
    
    \section{Initial Design}
    To come up with a initial design idea, brainstorming in multiple iterations will provide an early design that can be improved upon. The idea is to first brainstorm as many ideas as possible and then present them. Then, these ideas can be combined and improved and potentially be made into a better idea. In the end, a few select ideas will be chosen and voted on to get the best one.

    \section{Usability test}
    Once the the design is finalised, a usability test will be conducted in order to make sure that the interactions work as intended.
    
    \subsection{Sampling}
    Convenience sampling allows for an effective gather of participants, and will be used for usability testing\cite{bjoernerBog}.   
    
    \subsection{Observation}
    For the usability test, a Direct Observation method will be used\cite{bjoernerBog}. This is to observe the participants interactions within the game and to find out how they interact with an object. The purpose of this, is to find out if the participant knows how to interact or if somethings needs to be changed with the interaction.
    
    \subsection{System Usability Scale(SUS)}
    Along with the observations, the SUS will also be used to measure the usability of the prototype. SUS is a scale which consists of 10 questionnaire items with 5 responses ranging from Strongly Disagree to Strongly Agree. The results is gathered and given a SUS score which can be used to measure how good the usability in the prototype is\cite{SUS}. The questions can be found in appendix \ref{SUS}.
    

    \section{Final test}
     \subsection{Sampling}
     For the final test, the probability sampling method; Simple Random Sampling will be used\cite{bjoernerBog}. Each person in the target group population will have an equal chance to be selected.
     
     The test will be done using two different prototypes and the participants have the same probability of getting picked to use either one. 
    
    \subsection{To test for behaviour change in regards to meat consumption}
    Semi-Structured interview\cite{bjoernerBog}. A participant will be presented with either the intermixed version of the game or the overloaded one, they will not get the opportunity to try both as this might create a bias towards the results.
    "Did you feel like eating less burgers during the course of the game - if yes, why?"
    "What was your first impression of the game?"
    "What do you think the message of the game was?"
    "Would you consider eating less meat, after this experience?"
    "Did you feel that the message of the game came on too strong?"
    "Did you feel like you got a new perspective on meat consumption? -How so and how did it change?"
    "Did you reflect on your current standings in regards to meat consumption?"
    "After trying this game, do you feel any cognitive dissonance?"
    
    \subsection{Analysing the data}
    The analysis of the data will be done using either Meaning Condensation or Traditional Coding\cite{bjoernerBog}. Meaning Condensation takes what the test participant says in the interview and condenses the meaning down to only a few words or sentences. The condensed meanings will then be analysed and compared to what other participants answered. 
    
    \subsection{Test setup}
    There is gonna be two setups where a participant will either take part of an intermixed version of the prototype, or an overloaded one. After playing the game, the participant will then be asked the questions from the interview. Expected time per test is 20 minutes; 5 minutes for setup, 5 for playing the game, and 10 for discussion. Total amount of participants at least expected, is 20(10 per prototype).