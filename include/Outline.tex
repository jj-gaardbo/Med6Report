\chapter{Project Outline}
\section{Introduction}
    \subsection{UN Sustainable Development Goals}
    We all need to contribute not only politicians
    
\section{Climate change/Problem Area}\label{sec:problemArea}
    \subsection{Concito}
    What do people know and what do they do about it? 
    What are the consequences? 
    How does it affect them personally? 
    Are they aware vs. are they concerned?
    Why aren't they more concerned?\cite{concito}
    
    Do people rely on politicians to save the day.
    
    "Changes in household consumption, time use and urban form are crucial for a 1.5°C future"\cite{householdTimeUseCarbonFootprints}
    
    "General environmental concern or concern for the negative effects of air pollution appear not to motivate people to support programs designed to control global warming"\cite{climateConcernAndMotivationToChange}
    \todo{(F)Design Req: should revolve around peoples personal consumption}

    \subsection{Peoples attitude towards it}
    Problem with communicating about climate\cite{vrEngagementClimateChange}\cite{storyAboutClimateChange}.
    The climate change paradox\cite{the5Ds}.
    Dissonance in what people know and how they behave\cite{the5Ds}
    Psychological barriers:
    \begin{itemize}
        \item Distant
        \item Doom
        \item Dissonance
        \item Denial
        \item iDentity
    \end{itemize}
    \todo{(F)Design Req: should lower psychological barriers that make people have a higher concern.}
    
    How do you get people to be more engaged in climate change and change their behaviour\cite{storyAboutClimateChange}
    \todo{(NF)Design Req: should lower doom barrier by changing the climate change narrative to a positive one}

    
    Engagement is considered to have three components \cite{reorientingClimageChangeCommunication}\cite{vrEngagementClimateChange}
    \begin{itemize}
        \item understanding (knowledge) 
        \item emotion (interest and concern)
        \item behaviour (action)
    \end{itemize}
    \todo{(F)Design Req: should make people engaged in climate change and thereby change their behavior}

    \subsubsection{Sub-conclusion}
    In order to make people take action, we first need to make them engaged.
    - Make them understand the urgency of the issue
    - Speak to their emotions to raise their concern without being forceful and fear mongering (Positive narrative)
    - Make them learn by doing and encourage embodied understanding and cognition

\section{Behavior change}\label{sec:behaviourChange}
    This can be done in several ways
    
    \subsection{Transformational games}
    Transformational games are designed to change players. This framework is designed to bridge education with entertainment and create games that are both enriching and engaging\cite{transformationalFramework}.
    
    However, in order to not to be forceful and fear mongering(see \autoref{sec:problemArea}) when talking about serious issues "the embedded design model" could be used\cite{embeddedDesignModel}
    \begin{itemize}
        \item Intermixing (Balancing on-topic aspects with off-topic content)
        \item Obfuscating (Choice of genre, not presenting the message beforehand, delayed revelation)
        \item Distancing (Hypothetical, Metaphors, POX vs. Zombiepox)
    \end{itemize}
    
    Player engagement model?\cite{playerEngagement}
    
    \subsection{Immersive Virtual Environments (IVE)}
    Understanding: More conceptual understanding of complex issues\cite{vrEngagementClimateChange}
    Emotion: Potential to elicit emotional states through manipulating the visual, auditory and haptic stimuli presented to the user
    Behavior: Embodied cognition. We interact, communicate and learn through movements, gesture and physical activities\
    
    \subsection{Embodied cognition}
    "Objects and events that produce similar affective responses are remembered together in the same perceptual category, and these perceptual inputs are later used as bases in future judgments"\cite{ahn2011embodied}
    
    \subsection{Transformative VR}
    Transformative Framework:
    Persuasive experience => Cognitive shift => Behavioural change\cite{transformativeVR}
    "The overview effect"\cite{transformativeVR}
    
    "The added value provided by VR and AR in transforming our external experience by focusing on the high level of personal efficacy and self reflectiveness generated by their sense of presence and emotional engagement"\cite{riva2016transforming}.
    
    \subsection{Water consumption}
    Exaggerated ambient feedback has proved to have a positive effect.\cite{waterConsumption}
    
    \subsubsection{Sub-conclusion}
    Games on the subject of climate change are well-suited to address climate challenges because they can serve as effective tools for education and engagement\cite{gamesAsToolsForEngagement}
    The embedded model should be used to get the message across and lower the psychological barriers that arise when a serious social issue is communicated too directly and forthright(see \autoref{sec:problemArea})
    IVE's are effective visualisations tools to communicate complex issues.
    Transformative experiences are more accessible through VR

\section{Immersive virtual Reality}
    Virtual Reality could be used to effectively communicate about climate change. 
    The strength of the emotion(see \autoref{sec:problemArea}) is linked with the users sense of presence\cite{vrEngagementClimateChange}
    
    \subsection{Presence}
    "Newer forms of digital media such as video games or virtual reality simulation are particularly more likely to result in high presence compared to traditional media"\cite{ahn2011embodied}.
    
    Place Illusion (PI) is the type of presence that refers to the sense of "being there"\citep{vrImmersion}[p.~3551].
    
    Plausibility Illusion (Psi) is the illusion that what is perceived to be happening is actually happening\citep{vrImmersion}[p.~3553].
    
    The Plausibility Illusion does not not require realistic appearances, nor does it require realistic behaviour\citep{vrImmersion}[p.~3553]. In the control condition from Slater's paper, the virtual object simply had a humanoidal look, but could not be mistaken for a human in neither movements nor appearance\citep{vrImmersion}[p.~3553].
    
    The body is where PI and Psi are combined, and is instrumental in providing evidence for PI\citep{vrImmersion}[p.~3554]. When looking down at your body, wearing a HMD(Head Mounted Display), and seeing your body in the place that you expected provides strong evidence for PI\citep{vrImmersion}[p.~3554].
    
    You know that this virtual body is not yours, but simply a representation\citep{vrImmersion}[p.~3554]. When moving your body and seeing the limbs move accordingly, also provides a connection between the visual external world and the user\citep{vrImmersion}[p.~3554].
    
    \subsubsection{Sub-conclusion}
    By invoking a sense of presence, IVEs can support intense feelings that make a user think, feel and behave as though they are really embedded in the place represented by the computer generated virtual space.
    By not breaking place illusion and plausability illusion a user will feel presence

\section{Engaging stories}
    Engaging stories works great for knowledge transfer\citep{engagingStoryFundamentals}[p.~173-174].
    Stories simplify complex situations, most often by using typical characters with whom people can identify\cite{engagingStoryFundamentals}.

    Speaking to peoples emotions makes them engaged and changing their behaviour\cite{vrEngagementClimateChange}(See \autoref{sec:behaviourChange})
    Positive narratives can lower psychological barriers (Doom) (see \autoref{sec:problemArea})

\section{SOTA}

\section{FPS}

\section{Design Requirements}

\section{Methods}

\section{Design}

\section{Evaluation}

\section{Discussion}

\section{Conclusion}





\begin{comment}
\begin{enumerate}
    \item \textbf{Problem area}
        \begin{itemize}
            \item Target group (Danes)
            \begin{itemize}
                \item Specific issue/direction
                \item Target group
            \end{itemize}
            \begin{itemize}
                \item Survey target group
            \end{itemize}
        \end{itemize}

    \item \textbf{Analysis}
        \begin{itemize}
            \item Climate change games as tools for education and engagement\\
            “Recently, there has been a dramatic increase in the development of such games, many featuring innovative designs that blur traditional boundaries (for example, those that involve social media, alternative reality games, or those that involve direct action upon the real world). Here, we present an overview of the types of climate change game currently available, the benefits and trade-offs of their use, and reasons why they hold such promise for education and engagement regarding climate change.”\\
            https://www.nature.com/articles/nclimate2566\\
            \textbf{Conclusion: }Games on the subject of climate change are well-suited to address climate challenges because they can serve as effective tools for education and engagement Problem area/analysis?
            
            \item Reality is broken\\
            “In this groundbreaking book, she shows how we can leverage the power of games to fix what is wrong with the real world-from social problems like depression and obesity to global issues like poverty and climate change-and introduces us to cutting-edge games that are already changing the business, education, and nonprofit worlds. Written for gamers and non-gamers alike, Reality Is Broken shows that the future will belong to those who can understand, design, and play games.”\\
            %url{https://books.google.dk/books?hl=da&lr=&id=yiOtN_kDJZgC&oi=fnd&pg=PT10&dq=can+video+games+change+climate+change&ots=fjimVJaUZu&sig=7VqvHjDQzMYegHKUvuGHRaGHOgc&redir_esc=y#v=onepage&q=can%20video%20games%20change%20climate%20change&f=false}
            \textbf{Conclusion: }

            
            \item Using Interactive Technology to Support Students’ Understanding of the Greenhouse Effect and Global Warming\\
            “We discuss how students integrate their ideas about global climate change as a result of using virtual experiments that allow them to explore meaningful complexities of the climate system.”\\
            %url{https://link.springer.com/article/10.1007/s10956-011-9337-9}\\
            \textbf{Conclusion: } Middle-school students conduct virtual experiments to visualise and understand the consequences of climate change. Results show that participation increased the students understanding of the science. 

            
            \item Using Exaggerated Feedback in a Virtual Reality Environment to Enhance Behavior Intention of Water-Conservation\\
            “ANOVA was used to examine the effects of the experimental intervention. The results showed that the immersive virtual reality game in this study caused significant changes in cognition and behavior intention. This study contributes a novel persuasive technology specific to water resources.” \cite{waterConsumption}\\
            %url{https://psycnet.apa.org/record/2018-56404-016}
            \textbf{Conclusion: } Exaggerated feedback(EF) provides positive results in regards to behaviour change, when used in the form of ambient EF.
            The  experience  with ambient EF somehow  had a  lower level  of realism than that of the  no EF condition, but  it did not dramatically increase personal disbelief in the water conservation issue.
            Direct  EF  was  less  intuitive  for  participants  and  did  not  direct  their attention  to  the negative  consequences  of  their  behavior.
            \textbf{\textit{Supports embedded model article and no need for realism}}

        \end{itemize}
        
    \item \textbf{Learning/Persuasion/Behaviourism theory}
        \begin{itemize}
            \item "The Embedded Design Model"\cite{embeddedDesignModel}
                    \begin{itemize}
                        \item Intermixing (Balancing on-topic aspects with off-topic content)
                        \item Obfuscating (Choice of genre, not presenting the message beforehand, delayed revelation)
                        \item Distancing (Hypothetical, Metaphors, POX vs. Zombiepox)
                    \end{itemize}
            \item Change in behaviour
        \end{itemize}
        
        
    \item \textbf{SOTA}
        \begin{itemize}
            \item Immersion/Persuasive/Transformational games\cite{transformationalFramework}
            \begin{itemize}
                \item Barriers, behavior transformation, belief
            \end{itemize}
            \item Immersion/Narrative
            \item Mediums/VR
        \end{itemize}
        
        
    \item \textbf{Game titles}
         \begin{itemize}
            \item Board Games from the embedded model article\cite{embeddedDesignModel}
            \begin{itemize}
                \item POX vs. Zombie Pox
                \item Awkward  Moment
                \item The Luminists
                \item Buffalo: The Name Dropping Game
                \item Monarch
            \end{itemize}
            \item "Cops against manatees"
            \begin{itemize}
                \item The game featured climate related themes. 
                \item During testing, the game was shown to affect people’s recycling habits, improving them by 20 percent.
            \end{itemize}
         
            \item Job Sim
            \begin{itemize}
                \item Is in virtual reality
                \item Placed in a square where novelty interactions is the exciting gameplay mechanic
                \item Follows a "to-do" system to guide the player through the level
                \item A different take on making a boring task more fun
            \end{itemize}
            
            \item A new beginning
            \begin{itemize}
                \item Post apocalyptic story game - climate changes made the world go boom. 
                \item Excessive graphical scenes i.e. tornados over the Eiffel Tower.
            \end{itemize}
            
            \item Ciclania
            \begin{itemize}
                \item Climate change game where the protagonist have to reduce Co2 emissions
                \item Game is a 2D platformer where the protagonist uses time as a resource to use the environment to get through the level, i.e. making a tree grow really fast to climb it
            \end{itemize}
            
            \item Future Delta\cite{SOTA_FutureDelta}
            \begin{itemize}
                \item Teaches about local effects of climate change and what can be done in the local community
                \item Not in VR
                \item Takes place in a virtual reconstruction of the local area (Makes it relatable)
                \item \textit{Carbon Vision} highlights objects that are CO2 sinners
                \item Test methods are described in the article\cite{SOTA_FutureDelta}
            \end{itemize}
        \end{itemize}
        
    \item \textbf{Methods}\\
    Transformational framework
     \begin{itemize}
         \item Track player actions in the game\cite{transformationalFramework}
         \item Observe player actions in the real world\cite{transformationalFramework}
         \item Survey player intention and disposition\cite{transformationalFramework}
         \item Survey stakeholder impressions\cite{transformationalFramework}
         \item Test the player\cite{transformationalFramework}
         % chapter about Assessment
     \end{itemize}
     
     Embedded Design Model
     \begin{itemize}
         \item Intermixing: Testing if the on-topic content of the game is balanced to the off-topic content by doing tests that vary in the amount of on-topic aspects to see which is more effective.\cite{embeddedDesignModel}
         \item Obfuscating: Testing group of players already aware of the games message vs. groups who have no idea\cite{embeddedDesignModel}
         \item Distancing: Testing a "Realistic version" against a "Metaphorical version" to see which is most effective\cite{embeddedDesignModel}
     \end{itemize}
\end{enumerate}
\end{comment}
