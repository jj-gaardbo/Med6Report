\chapter{Project Outline}\todo{Outcomment this chapter in Main.tex later}
\section{Introduction}
    \subsection{UN Sustainable Development Goals}
    We all need to contribute not only politicians
    
\section{Climate change/Problem Area}\label{sec:problemArea}
    \subsection{Concito}
    What do people know and what do they do about it? 
    What are the consequences? 
    How does it affect them personally? 
    Are they aware vs. are they concerned?
    Why aren't they more concerned?\cite{concito}
    
    Do people rely on politicians to save the day.
    
    "Changes in household consumption, time use and urban form are crucial for a 1.5°C future"\cite{householdTimeUseCarbonFootprints}
    
    "General environmental concern or concern for the negative effects of air pollution appear not to motivate people to support programs designed to control global warming"\cite{climateConcernAndMotivationToChange}
    \todo{(F)Design Req: should revolve around peoples personal consumption}

    \subsection{Peoples attitude towards it}
    Problem with communicating about climate\cite{vrEngagementClimateChange}\cite{storyAboutClimateChange}.
    The climate change paradox\cite{the5Ds}.
    Dissonance in what people know and how they behave\cite{the5Ds}
    Psychological barriers:
    \begin{itemize}
        \item Distant
        \item Doom
        \item Dissonance
        \item Denial
        \item iDentity
    \end{itemize}
    \todo{(F)Design Req: should lower psychological barriers that make people have a higher concern.}
    
    How do you get people to be more engaged in climate change and change their behaviour\cite{storyAboutClimateChange}
    \todo{(NF)Design Req: should lower doom barrier by changing the climate change narrative to a positive one}

    
    Engagement is considered to have three components \cite{reorientingClimageChangeCommunication}\cite{vrEngagementClimateChange}
    \begin{itemize}
        \item understanding (knowledge) 
        \item emotion (interest and concern)
        \item behaviour (action)
    \end{itemize}
    \todo{(F)Design Req: should make people engaged in climate change and thereby change their behavior}

    \subsubsection{Sub-conclusion}
    In order to make people take action, we first need to make them engaged.
    - Make them understand the urgency of the issue
    - Speak to their emotions to raise their concern without being forceful and fear mongering (Positive narrative)
    - Make them learn by doing and encourage embodied understanding and cognition

\section{Behavior change}\label{sec:behaviourChange}
    This can be done in several ways
    
    \subsection{Transformational games}
    Transformational games are designed to change players. This framework is designed to bridge education with entertainment and create games that are both enriching and engaging\cite{transformationalFramework}.
    
    However, in order to not to be forceful and fear mongering(see \autoref{sec:problemArea}) when talking about serious issues "the embedded design model" could be used\cite{embeddedDesignModel}
    \begin{itemize}
        \item Intermixing (Balancing on-topic aspects with off-topic content)
        \item Obfuscating (Choice of genre, not presenting the message beforehand, delayed revelation)
        \item Distancing (Hypothetical, Metaphors, POX vs. Zombiepox)
    \end{itemize}
    
    Player engagement model?\cite{playerEngagement}
    
    \subsection{Immersive Virtual Environments (IVE)}
    Understanding: More conceptual understanding of complex issues\cite{vrEngagementClimateChange}
    Emotion: Potential to elicit emotional states through manipulating the visual, auditory and haptic stimuli presented to the user
    Behavior: Embodied cognition. We interact, communicate and learn through movements, gesture and physical activities\
    
    \subsection{Embodied cognition}
    "Objects and events that produce similar affective responses are remembered together in the same perceptual category, and these perceptual inputs are later used as bases in future judgments"\cite{ahn2011embodied}
    
    \subsection{Transformational games in VR}
    Transformative Framework:
    Persuasive experience => Cognitive shift => Behavioural change\cite{transformativeVR}
    "The overview effect"\cite{transformativeVR}
    
    "The added value provided by VR and AR in transforming our external experience by focusing on the high level of personal efficacy and self reflectiveness generated by their sense of presence and emotional engagement"\cite{riva2016transforming}.
    
    \subsection{Water consumption}
    Exaggerated ambient feedback has proved to have a positive effect.\cite{waterConsumption}
    
    \subsubsection{Sub-conclusion}
    Games on the subject of climate change are well-suited to address climate challenges because they can serve as effective tools for education and engagement\cite{gamesAsToolsForEngagement}
    The embedded model should be used to get the message across and lower the psychological barriers that arise when a serious social issue is communicated too directly and forthright(see \autoref{sec:problemArea})
    IVE's are effective visualisations tools to communicate complex issues.
    Transformative experiences are more accessible through VR

\section{Immersive virtual Reality}
    Virtual Reality could be used to effectively communicate about climate change. 
    The strength of the emotion(see \autoref{sec:problemArea}) is linked with the users sense of presence\cite{vrEngagementClimateChange}
    
    \subsection{Presence}
    "Newer forms of digital media such as video games or virtual reality simulation are particularly more likely to result in high presence compared to traditional media"\cite{ahn2011embodied}.
    
    Place Illusion (PI) is the type of presence that refers to the sense of "being there"\citep{vrImmersion}[p.~3551].
    
    Plausibility Illusion (Psi) is the illusion that what is perceived to be happening is actually happening\citep{vrImmersion}[p.~3553].
    
    The Plausibility Illusion does not not require realistic appearances, nor does it require realistic behaviour\citep{vrImmersion}[p.~3553]. In the control condition from Slater's paper, the virtual object simply had a humanoidal look, but could not be mistaken for a human in neither movements nor appearance\citep{vrImmersion}[p.~3553].
    
    The body is where PI and Psi are combined, and is instrumental in providing evidence for PI\citep{vrImmersion}[p.~3554]. When looking down at your body, wearing a HMD(Head Mounted Display), and seeing your body in the place that you expected provides strong evidence for PI\citep{vrImmersion}[p.~3554].
    
    You know that this virtual body is not yours, but simply a representation\citep{vrImmersion}[p.~3554]. When moving your body and seeing the limbs move accordingly, also provides a connection between the visual external world and the user\citep{vrImmersion}[p.~3554].
    
    \subsubsection{Sub-conclusion}
    By invoking a sense of presence, IVEs can support intense feelings that make a user think, feel and behave as though they are really embedded in the place represented by the computer generated virtual space.
    By not breaking place illusion and plausability illusion a user will feel presence

\section{Engaging stories}
    Engaging stories works great for knowledge transfer\citep{engagingStoryFundamentals}[p.~173-174].
    Stories simplify complex situations, most often by using typical characters with whom people can identify\cite{engagingStoryFundamentals}.

    Speaking to peoples emotions makes them engaged and changing their behaviour\cite{vrEngagementClimateChange}(See \autoref{sec:behaviourChange})
    Positive narratives can lower psychological barriers (Doom) (see \autoref{sec:problemArea})

\section{SOTA}

\section{FPS}

\section{Design Requirements}

\section{Methods}

\section{Design}

\section{Evaluation}

\section{Discussion}

\section{Conclusion}










\begin{comment}




Evidence for Learning through Immersion:
\url{http://science.sciencemag.org/content/323/5910/66}

Evidence for Virtual Reality as immersive tool:
"Demonstrations of high-end VR facilities clearly show, however, that immersive VR works."
\url{http://www.cs.rug.nl/~roe/courses/OriInf/Bowman-Virtual-Reality}

Evidence that Immersive VR is good for learning
\url{https://link.springer.com/article/10.1007/BF00896880}

Games can change attitude and behaviours:
"3. Associate - games can help us make new mental associations and possibly help us save the planet.

Flanagan’s team created a “Cards against humanity” type of game, but with an environmental message, called “Cops against manatees.” The game featured climate related themes. During testing, the game was shown to affect people’s recycling habits, improving them by 20 percent."

\url{https://bigthink.com/paul-ratner/video-games-can-have-a-meaningful-social-impact}

\section{Climate change}
    % Peoples attitude towards it and 
    % How do they behave
    % Psychological barriers
    
    \begin{itemize}
        \item The problem of Morale Corruption\\
            “This convergence justifies calling it a 'perfect moral storm'. One consequence of this storm is that, even if the other difficult ethical questions surrounding climate change could be answered, we might still find it difficult to act. For the storm makes us extremely vulnerable to moral corruption.”\\
            https://www.jstor.org/stable/30302196?seq=1\#page\_scan\_tab\_contents\\
            \textbf{Conclusion: } Climate change caused by many individuals, limits response from person (Fragmentation of Agency). Dispersion of causes and effect, the as the response of the climate change is slow and the effect begins small, the lack of phenomenons makes it appear as less of a problem. PROBLEM AREA
            
            \item Predictors of public climate change awareness and risk perception around the world\\
            “The results suggest that improving basic education, climate literacy, and public understanding of the local dimensions of climate change are vital to public engagement and support for climate action.”\\
            https://www.nature.com/articles/nclimate2728\\
            \textbf{Conclusion: } PROBLEM AREA
            
            \item In what sense does the public need to understand global climate change?\\
            “General environmental concern or concern for the negative effects of air pollution appear not to motivate people to support programs designed to control global warming.”\\
            https://journals.sagepub.com/doi/10.1088/0963-6625/9/3/301\\
            \textbf{Conclusion: } PROBLEM AREA
            
            \item Climate change-induced migration and violent conflict\\
            “People can adapt to these problems by staying in place and doing nothing, staying in place and mitigating the problems, or leaving the affected areas. The choice between these options will depend on the extent of problems and mitigation capabilities. People living in lesser developed countries may be more likely to leave affected areas, which may cause conflict in receiving areas. My findings support this theory, and suggest certain policy implications for climate change.”\\
            https://www.sciencedirect.com/science/article/pii/S0962629807000601\\
            \textbf{Conclusion: } Problem area empowerment
            
            \item Rethinking climate communications and the psychological climate paradox (The 5Ds)
                \begin{itemize}
                    \item Distant => Feels personal, near and urgent.
                    \item Doom => Uses cognitive framings that do not backfire on the climate issue through negative affects.
                    \item Dissonance => Reduces dissonance by providing opportunities for visible and consistent action.
                    \item Denial => Avoids triggering the emotional need for denial.
                    \item iDentity => Reduces cultural and political polarization on the issue.
                \end{itemize}
                \textbf{Conclusion: } These barriers need to be lowered. And later sections in the Analysis are going to do that.
            \cite{the5Ds}
            
            \item What we think about when we're trying not to think about climate change
            The more facts that pile up about global warming, the greater the resistance to them grows, making it harder to enact measures to reduce greenhouse gas emissions and prepare communities for the inevitable change ahead. It is a catch-22 that starts from an inadequate understanding of the way most humans think, act, and live in the world around them. With dozens of examples―from the private sector to government agencies―the book shows how to retell the story of climate change and, at the same time, create positive, meaningful actions that can be supported even by deniers.\cite{storyAboutClimateChange}
            \textbf{Conclusion: } Make the stories about climate change positive, near, personal, and urgent.
    \end{itemize}

    

\section{Behavior change}
\textit{Thus, the results of Study 1 suggested that presence and personal
control may be the processes through which embodied experiences influence selfefficacy and pro-environmental behavior.}

Self-reflectiveness, personal efficiacy
    % How to change behavior, and in particular how to change climate behavior.
    % 
    
    % The VR aspects of this article will be discussed in a later section, but here we can write about the difficulty of communicating the issue and not mention the VR part
    Climate change is an issue known and acknowledged by most people, but they have a hard time communicating about causes and consequences. The effects seem remote and invoking fear is counterproductive. \textit{"Engagement is considered to have three components 1) understanding(knowledge), 2) emotion (interest and concern), and 3) behaviour (action). Recent attention has turned to the potential affordances of immersive virtual environments (IVEs) – including virtual environments, virtual reality (VR), augmented reality (AR) and mixed reality (MR) - to support public engagement with climate change"}\cite{vrEngagementClimateChange}
    
    "An attitude consists of two components that we can shape (through gameplay):the cognitive(beliefs) and the affective(feelings). “An attitude is a combination of what you believe or expect of a certain object and how you feel about (evaluate) these expectations.”\cite{persuasiveGameplay}
    
    
    "The Embedded Design Model"\cite{embeddedDesignModel}
    \begin{itemize}
        \item Intermixing (Balancing on-topic aspects with off-topic content)
        \item Obfuscating (Choice of genre, not presenting the message beforehand, delayed revelation)
        \item Distancing (Hypothetical, Metaphors, POX vs. Zombiepox)
    \end{itemize}
    \cite{transformationalFramework}
    
    Persuasive experience => Cognitive shift => Behavioural change
    \cite{transformativeVR}


\section{Engaging stories}
    Engaging stories works great for knowledge transfer\citep{engagingStoryFundamentals}[p.~173-174].
    Stories simplify complex situations, most often by using typical characters with whom people can identify\cite{engagingStoryFundamentals}.
    
    A character is something that a reader can identity with, and that they can care and follow through the story\citep{engagingStoryFundamentals}[p.~181].
    
    Engaging story needs to be concrete, using examples for example, or concrete numbers and figures \citep{engagingStoryFundamentals}[p.~181-182].
    
    Stories that challenge previous knowledge like a story about the life cycle environment impacts of plastic versus paper bags, where plastic bags are not always more environmentally damaging than paper bags\citep{engagingStoryFundamentals}[p.~183].
    
\section{Immersion Virtual Reality}
    Immersive Virtual Environments and Climate Change Engagement
    Climate Change will affect our lives, yet communicating about climate change has proven to be more complicated than initially thought. Recent attention has turned to the potential affordances of immersive virtual environments (IVE) to support public engagement around climate change\cite{vrEngagementClimateChange}.
    \textbf{Conclusion: } How Immersive Virtual reality could be used to effectively communicate about climate change

    Place Illusion (PI) is the type of presence that refers to the sense of "being there"\citep{vrImmersion}[p.~3551].
    
    Plausibility Illusion (Psi) is the illusion that what is perceived to be happening is actually happening\citep{vrImmersion}[p.~3553].
    
    The Plausibility Illusion does not not require realistic appearances, nor does it require realistic behaviour\citep{vrImmersion}[p.~3553]. In the control condition from Slater's paper, the virtual object simply had a humanoidal look, but could not be mistaken for a human in neither movements nor appearance\citep{vrImmersion}[p.~3553].
    
    The body is where PI and Psi are combined, and is instrumental in providing evidence for PI\citep{vrImmersion}[p.~3554]. When looking down at your body, wearing a HMD(Head Mounted Display), and seeing your body in the place that you expected provides strong evidence for PI\citep{vrImmersion}[p.~3554].
    
    You know that this virtual body is not yours, but simply a representation\citep{vrImmersion}[p.~3554]. When moving your body and seeing the limbs move accordingly, also provides a connection between the visual external world and the user\citep{vrImmersion}[p.~3554].

\end{comment}