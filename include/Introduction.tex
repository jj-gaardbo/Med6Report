\chapter{Introduction}\label{chap:introduction}

The Danes are concerned about the climate, it is all over the news and radio, and it is one of the big focal points of the upcoming election. Not enough is done, and what is done is happening too slowly, the people living in Denmark need to translate  their climate change concern into climate actions.


%Climate change (Overordnet om klima forandringer)

%UN SDGs

%Concern

%Societal / Personal responsibility

%Behavior

%Learning / Persuasion / Mediation

%Transformational Games / Games for Change

\begin{itemize}
    \item 88 percent of Danes think that global warming is a serious issue\cite{concito}.
    \item 80 percent of 30-39 years old think that global warming is a serious issue\cite{concito}.
    \item 54 percent think it is necessary to change lifestyle to improve the climate\cite{concito}.
    \item 48 percent are worried that climate changes will affect them personally\cite{concito}.
    \item 47 percent says they have done something actively to improve the climate\cite{concito}.
    \item 12 percent find it easy to know what to buy in grocery shops that is good for the climate\cite{concito}.
    \item What makes people think environment is a problem, but does not do anything about it
\end{itemize}

\section{Initial Problem Statement}
    How can a digital game raise the level of concern about climate change in an individual user?
    
    