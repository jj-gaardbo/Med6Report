\chapter{Discussion}
This chapter will go into detail and discuss the results presented in evaluation. 

\section{Test result discussion}
This section will discuss each part of the results gathered and debate why they are like they are, and how they could be improved or changed.

     \subsection{Enjoyment and engagement}
     The overloaded version of the game had a higher level of enjoyment than the intermixed one (5,1 compared to 4,6). The reason for this could be that the overloaded version had more visually appealing elements, which means that there is more to look at while listening to the radio. This also correlates with neither the participants of the intermixed version nor the participants of the overloaded version found the message coming on too strong. However, this could also be a coincidence because the sample size is quite low (n=10) to have a statistical relevance. In general enjoyment were for both versions on the positive side of the scale which means that the participants enjoyed the experience. To get a higher enjoyment, more interaction elements should be added to the game. Multiple participants expressed that the burger were the only interaction element and they wish there were more to do in the game. Furthermore, adding more visual elements for the user to observe while using the prototypes could also add to the enjoyment of the game.
     
     Both versions of the prototype have the exact same engagement value of 4,6. This can be considered positive as a user using either of the prototypes can be considered equally engaged. This also means that by not having a large different engagement value, it can be considered to not create a difference between both versions when considering engagement as a factor when discussing other results. However, like enjoyment, due to the small sample size the results can not be trusted completely but can only be seen as an example. Like enjoyment, engagement could also be affected by adding more engaging interaction elements.
     
     \subsection{Major themes}
     \subsubsection*{Message of the game}
     Depending on which version a participant tried they perceived the message of the game differently. The overloaded version, participants easily understood that meat consumption were the message, where in the intermixed version, climate change were the perceived message. The ove
     
     \subsection{View}
     
\section{Process discussion}
    The test setup, using an anechoic chamber for isolating the test participant, a moderator for controller the flow of the test, and an observer taking notes. The process of taking notes during the test, led to some subject were interpreted, or missed during the interview.
    
    \subsection{Reliability}
        To ensure reliability, the setup was planned, as seen in \autoref{fig:final_test_setup}, and following the semi structured interview structure, allowed for each interview to have a similar flow and structure.
    \subsection{Validity}
        