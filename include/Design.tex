\chapter{Design}

\section{Initial Design}
This section will present the initial concept ideas, and explain how they fulfil the design requirements found in the analysis. Finally, the concept for the final design will be chosen.

\subsection{Concepts}

\begin{itemize}
    \item Get-To-Work-Simulator:\\
    In "Get-To-Work-Simulator", the user will try to get to work as efficiently as possible, through a number of options presented throughout the experience. The user will play through mini-games, and decide how to get from A to B. Get-To-Work-Simulator will focus on improving how the individual gets about in their daily lives, and try to get more people to choose bicycle or public transportation, which the analysis explained is a significant cause for pollution.\\
    INSERT GET-TO-WORK-SIMULATOR PICTURE HERE\\
    \item House:\\
    In "House", the users finds themselves in a regular house. As they start their day, they will be presented with a series of tasks for them complete. These tasks are inspired by the mechanics found in the popular VR-game "Job simulator", however they are changed to be relevant to climate change.\\
    INSERT HOUSE PICTURE HERE\\
    \item Local-Community-God-Simulator:\\
    Local-Community-God-Simulator
    \item 60 seconds climate saver:\\
    60 seconds climate saver
    \item Narrative Driver:\\
    Narrative Driver
    \item Making Mars Great Again:\\
    Making Mars Great Again.
\end{itemize}

\subsection{Choosing the concept}
Narrative driver is chosen because nice.

\section{Final design}
Narrative driver explained overall

\subsection{Narration}

\subsection{Graphics}

\subsection{Environment}

\subsection{Interactions and controls - Low fidelity prototype}

\section{Usability test}

